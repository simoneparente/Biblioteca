\chapter{Requisiti identificati}
Si vuole sviluppare un sistema informativo di gestione di una biblioteca digitale contenente \textbf{Libri} e
\textbf{Articoli scientifici}.

I libri possono essere \textbf{Didattici} o \textbf{Romanzi}.

In particolare, questi ultimi possono essere parte di \textbf{Collane}, raggruppate per caratteristiche
comuni, e appartenere ad una \textbf{Serie} se hanno uno o più seguiti, gli articoli possono essere parte
di una \textbf{Rivista} oppure essere presentati durante una \textbf{Conferenza}.

Il sistema dovrà inoltre permettere ad un \textbf{Utente} la ricerca di un libro
e recuperare la lista di \textbf{Negozi} in cui sia possibile acquistare quest'ultimo.
L'utente potrà inoltre ricercare una serie (o collana) di libri e un negozio in cui
quest'ultima potrà essere acquistata nel caso in cui al momento della ricerca non ci fosse alcun negozio idoneo,
l'utente potrà inoltrare una \textbf{Richiesta} di notifica nel momento in cui uno dei negozi avrà tutti i libri 
appartenenti alla serie.

In particolare sono state identificate le seguenti entità:
\begin{enumerate}
    \item \textbf{Pubblicazione}: Generalizzazione di un libro o un articolo scientifico
    \item \textbf{Libro}: Specializzazione di una Pubblicazione
    \item \textbf{Articolo scientifico}: Specializzazione di una Pubblicazione
    \item \textbf{Rivista}: Entità che identifica un insieme di articoli
    \item \textbf{Evento}: Generalizzazione di una Conferenza o di una Presentazione
    \item \textbf{Conferenza}: Specializzazione di un Evento
    \item \textbf{Presentazione}: Specializzazione di un Evento
    \item \textbf{Autore}: Entità che identifica l'autore di un Libro o di un Articolo
    \item \textbf{Negozio}: Entità che identifica un Negozio
    \item \textbf{Serie}: Entità che identifica un insieme di libri con caratteristiche simili
    \item \textbf{Richiesta}: Entità che identifica la richiesta di disponibilità di una serie da 
    parte di un utente
\end{enumerate}