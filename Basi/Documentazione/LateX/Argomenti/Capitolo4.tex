\chapter{Schema Fisico}

In questo ultimo capitolo esamineremo i meccanismi necessari per la traduzione di uno schema logico 
in uno schema fisico. Andremo a definire le tabelle con i relativi attributi e tipi dei dati, le
funzioni, le procedure, i trigger e i vincoli. Con questi elementi, sarà possibile creare un database
relazionale con una struttura specifica che soddisfi i requisiti identificati nel Capitolo \ref{Capitolo1}.

\section{Creazione Tabelle}
\subsection{Tabella Articoli}
\begin{lstlisting}
    CREATE TABLE b.Articoli
    ID_Articolo       SERIAL,
    Titolo            VARCHAR(128),
    DOI               VARCHAR(128),
    DataPubblicazione DATE,
    Disciplina        VARCHAR(128),
    Editore           VARCHAR(128),
    Lingua            VARCHAR(128),
    Formato           VARCHAR(128),

    CONSTRAINT PK_Articoli PRIMARY KEY (ID_Articolo),
    CONSTRAINT UK_Articolo UNIQUE (DOI);
\end{lstlisting}
\subsection{Tabella Autore}
\begin{lstlisting}
    CREATE TABLE b.Autore
    ID_Autore SERIAL,
    Nome      VARCHAR(128),
    Cognome   VARCHAR(128),

    CONSTRAINT PK_Autore PRIMARY KEY (ID_Autore);
\end{lstlisting}
\subsection{Tabella AutoreArticolo}
\begin{lstlisting}
    CREATE TABLE b.AutoreArticolo
    ID_Autore   SERIAL,
    ID_Articolo SERIAL,

    CONSTRAINT PK_AutoreArticolo PRIMARY KEY (ID_Autore, ID_Articolo),
    CONSTRAINT FK_AutoreArticolo_Autore FOREIGN KEY (ID_Autore) REFERENCES b.Autore (ID_Autore) ON DELETE CASCADE,
    CONSTRAINT FK_AutoreArticolo_Articoli FOREIGN KEY (ID_Articolo) REFERENCES b.Articoli (ID_Articolo) ON DELETE CASCADE;
\end{lstlisting}

\subsection{Tabella Riviste}
\begin{lstlisting}
    CREATE TABLE b.Riviste
    ID_Rivista        SERIAL,
    ISSN              VARCHAR(128),
    Nome              VARCHAR(128),
    Argomento         VARCHAR(128),
    DataPubblicazione DATE,
    Responsabile      VARCHAR(128),
    Prezzo            FLOAT,

    CONSTRAINT PK_Riviste PRIMARY KEY (ID_Rivista);
\end{lstlisting}

\subsection{Tabella ArticoliInRiviste}
\begin{lstlisting}
CREATE TABLE b.ArticoliInRiviste
    ID_Articolo SERIAL,
    ID_Rivista  SERIAL,

    CONSTRAINT PK_ArticoliInRiviste PRIMARY KEY (ID_Articolo, ID_Rivista),
    CONSTRAINT FK_ArticoliInRiviste_Articolo FOREIGN KEY (ID_Articolo) REFERENCES b.Articoli (ID_Articolo) ON DELETE CASCADE,
    CONSTRAINT FK_ArticoliInRiviste_Rivista FOREIGN KEY (ID_Rivista) REFERENCES b.Riviste (ID_Rivista) ON DELETE CASCADE
\end{lstlisting}

\newpage

\subsection{Tabella Evento}
\begin{lstlisting}
    ID_Evento          SERIAL,
    Nome               VARCHAR(128),
    Indirizzo          VARCHAR(128),
    StrutturaOspitante VARCHAR(128),
    DataInizio         DATE,
    DataFine           DATE,
    Responsabile       VARCHAR(128),

    CONSTRAINT PK_Evento PRIMARY KEY (ID_Evento),
    CONSTRAINT CK_Date CHECK (DataInizio <= DataFine),
    CONSTRAINT UK_Evento UNIQUE (Indirizzo, StrutturaOspitante, DataInizio, DataFine, Responsabile);
\end{lstlisting}

\subsection{Conferenza}
\begin{lstlisting}
    CREATE TABLE b.Conferenza
    ID_Articolo SERIAL,
    ID_Evento   SERIAL,

    CONSTRAINT PK_Conferenza PRIMARY KEY (ID_Articolo, ID_Evento),
    CONSTRAINT FK_Conferenza_Articolo FOREIGN KEY (ID_Articolo) REFERENCES b.Articoli (ID_Articolo) ON DELETE CASCADE,
    CONSTRAINT FK_Conferenza_Evento FOREIGN KEY (ID_Evento) REFERENCES b.Evento (ID_Evento) ON DELETE CASCADE;
\end{lstlisting}

\subsection{Tabella Libri}
\begin{lstlisting}
    CREATE TABLE b.Libri
    ID_Libro          SERIAL,
    Titolo            VARCHAR(128),
    ISBN              VARCHAR(128),
    DataPubblicazione DATE,
    Editore           VARCHAR(128),
    Genere            VARCHAR(128),
    Lingua            VARCHAR(128),
    Formato           VARCHAR(128),
    Prezzo            FLOAT,

    CONSTRAINT PK_Libri PRIMARY KEY (ID_Libro),
    CONSTRAINT UK_Libro UNIQUE (ISBN),
    CONSTRAINT CK_Libri CHECK (Prezzo > 0),
    CONSTRAINT CK_Titolo (Titolo IS NOT NULL);
\end{lstlisting}

\subsection{Tabella AutoreLibro}
\begin{lstlisting}
    CREATE TABLE b.AutoreLibro
    ID_Autore SERIAL,
    ID_Libro  SERIAL,

    CONSTRAINT PK_AutoreLibro PRIMARY KEY (ID_Autore, ID_Libro),
    CONSTRAINT FK_AutoreLibro_Autore FOREIGN KEY (ID_Autore) REFERENCES b.Autore (ID_Autore) ON DELETE CASCADE,
    CONSTRAINT FK_AutoreLibro_Libro FOREIGN KEY (ID_Libro) REFERENCES b.Libri (ID_Libro) ON DELETE CASCADE;
\end{lstlisting}

\subsection{Tabella Presentazione}
\begin{lstlisting}
    CREATE TABLE b.Presentazione
    ID_Evento SERIAL,
    ID_Libro  SERIAL,

    CONSTRAINT PK_Presentazione PRIMARY KEY (ID_Evento, ID_Libro),
    CONSTRAINT FK_Presentazione_Evento FOREIGN KEY (ID_Evento) REFERENCES b.Evento (ID_Evento) ON DELETE CASCADE,
    CONSTRAINT FK_Presentazione_Libro FOREIGN KEY (ID_Libro) REFERENCES b.Libri (ID_Libro) ON DELETE CASCADE;
\end{lstlisting}

\subsection{Tabella Serie}
\begin{lstlisting}
    CREATE TABLE b.Serie
    ID_Serie SERIAL,
    ISSN     VARCHAR(128),
    Nome     VARCHAR(128),

    CONSTRAINT PK_Serie PRIMARY KEY (ID_Serie),
    CONSTRAINT UK_Serie UNIQUE (ISSN);
\end{lstlisting}

\newpage

\subsection{Tabella LibriInSerie}
\begin{lstlisting}
    CREATE TABLE b.LibriInSerie
    ID_Serie INTEGER,
    ID_Libro INTEGER,

    CONSTRAINT PK_LibriInSerie PRIMARY KEY (ID_Serie, ID_Libro),
    CONSTRAINT FK_Libri_Serie FOREIGN KEY (ID_Serie) REFERENCES b.Serie (ID_Serie) ON DELETE CASCADE,
    CONSTRAINT FK_Serie_Libri FOREIGN KEY (ID_Libro) REFERENCES b.Libri (ID_Libro) ON DELETE CASCADE;
\end{lstlisting}

\subsection{Tabella Negozio}
\begin{lstlisting}
    CREATE TABLE b.Negozio
    ID_Negozio SERIAL,
    Nome       VARCHAR(128),
    Tipo       VARCHAR(128),

    CONSTRAINT PK_Negozio PRIMARY KEY (ID_Negozio);
\end{lstlisting}

\subsection{Tabella Stock}
\begin{lstlisting}
    CREATE TABLE b.Negozio
    ID_Negozio SERIAL,
    Nome       VARCHAR(128),
    Tipo       VARCHAR(128),

    CONSTRAINT PK_Negozio PRIMARY KEY (ID_Negozio);
\end{lstlisting}

\subsection{Tabella Utente}
\begin{lstlisting}
    CREATE TABLE b.Utente
    ID_Utente SERIAL,
    Username  VARCHAR(128),
    Password  VARCHAR(128),
    Permessi  b.TipoUtente DEFAULT '0',

    CONSTRAINT PK_Utente PRIMARY KEY (ID_Utente),
    CONSTRAINT UK_Utente UNIQUE (Username);
\end{lstlisting}

\newpage

\subsection{Tabella Richiesta}
\begin{lstlisting}
    CREATE TABLE b.Richiesta
    ID_Utente     SERIAL,
    ID_Serie      SERIAL,

    CONSTRAINT PK_Richiesta PRIMARY KEY (ID_Utente, ID_Serie),
    CONSTRAINT FK_Richiesta_Utente FOREIGN KEY (ID_Utente) REFERENCES b.Utente (ID_Utente) ON DELETE CASCADE,
    CONSTRAINT FK_Richiesta_Serie FOREIGN KEY (ID_Serie) REFERENCES b.Serie (ID_Serie) ON DELETE CASCADE;
\end{lstlisting}

\section{Creazione Funzioni e Procedure}