\documentclass{report}
\usepackage[utf8]{inputenc}
\usepackage{booktabs}
\usepackage{graphicx}
\usepackage[italian]{babel}
\usepackage{xcolor}
\usepackage{hyperref}
\usepackage{longtable}
\usepackage{listings}
\usepackage{color}
\UseRawInputEncoding
\definecolor{dkgreen}{rgb}{0,0.6,0}
\definecolor{gray}{rgb}{0.5,0.5,0.5}
\definecolor{mauve}{rgb}{0.58,0,0.82}
\lstset{language=SQL,
  basicstyle={\small\ttfamily},
  belowskip=3mm,
  breakatwhitespace=true,
  breaklines=true,
  classoffset=0,
  columns=flexible,
  commentstyle=\color{dkgreen},
  framexleftmargin=0.25em,
  frameshape={}{yy}{}{}, %To remove to vertical lines on left, set `frameshape={}{}{}{}`
  keywordstyle=\color{blue},
  numbers=left, %If you want line numbers, set `numbers=left`
  numberstyle=\tiny\color{gray},
  showstringspaces=false,
  stringstyle=\color{mauve},
  tabsize=3,
  xleftmargin =1em
}


\begin{document}
\begin{sloppypar}
    \begin{figure}[htbp!]
        \begin{center}
            \includegraphics[width=.25\textwidth]{Immagini/FedericoII.png}
        \end{center}
    \end{figure}
    
    {\scshape\Large\bfseries Inserire qui il titolo}
    
    \begin{center}
        Inserire il nome \\ Inserire la matricola \\ Inserire la data
    \end{center}
    
    \newpage
    
    \tableofcontents
    
    \chapter{Requisiti identificati}
Si vuole sviluppare un sistema informativo di gestione di una biblioteca digitale contenente \textbf{Libri} e
\textbf{articoli scientifici}.

I libri possono essere \textbf{Didattici} o \textbf{Romanzi}

In particolare, questi ultimi possono essere parte di \textbf{Collane}, raggruppate per caratteristiche
comuni, e appartenere ad una \textbf{Serie} se hanno uno o più seguiti, gli articoli possono essere parte
di una \textbf{Rivista} oppure essere presentati durante una \textbf{Conferenza}.

Il sistema dovrà inoltre permettere ad un \textbf{Utente} la ricerca di una serie
e recuperare la lista di \textbf{Negozi} in cui sia possibile acquistare quest'ultima per intero,
nel caso in cui al momento della ricerca non ci fosse alcun negozio idoneo, l'utente potrà richiedere
di essere notificato nel momento in cui uno dei negozi avrà tutti i libri appartenenti alla serie.
    \chapter{Progettazione concettuale}
    \section{Class Diagram}

    \includegraphics[scale=0.3]{Immagini/UML_v1_0.png}
        
    \section{Analisi della ristrutturazione del Class Diagram}
        In questa fase verranno effettueremo delle modifiche che renderanno il Class Diagram
        più adatto a una traduzione al modello logico \textcolor{red}{(magari scriviamo meglio sta parte)}
        \subsection{Analisi delle ridondanze}
        \textcolor{red}{(non saprei)}
        \subsection{Analisi degli identificativi}
        In questa fase andremo a scegliere uno o più attributi atti a identificare univocamente
        le varie entità presenti nello schema precedente, in particolare:
            \begin{enumerate}
            \item L'entità \textbf{Libro} presenta l'attributo ISBN che rappresenta una possibile chiave primaria,
                  tuttavia è stato scelto di aggiungere un attributo \textit{ID\_Libro} in modo tale da aumentare
                  la velocità di accesso agli indici.
            \item Per \textbf{Articolo scientifico} la situazione è analoga, è stato quindi aggiunto un attributo
                  \textit{ID\_Articolo}.
            \item Nel caso dell'entità \textbf{Rivista}, la quale presenta un attributo ISSN che è chiave candidata,
                  di inserire un ulteriore attributo \textit{ID\_Rivista}.
            \item Sarebbe possibile identificare un \textbf{Evento} tramite un insieme piuttosto ampio di attributi, è
                  stato quindi aggiunto un attributo \textit{ID\_Evento}
            \item \textbf{Autore}: 
            \item \textbf{Negozio}:
            \item \textbf{Serie}: 
            \end{enumerate}
        \subsection{Rimozione degli attributi multipli}
            
        \subsection{Rimozione degli attributi composti}
            
        \subsection{Partizione/Accorpamento delle associazioni}
            
        \subsection{Rimozione delle gerarchie}
    
    \section{Class Diagram ristrutturato}
    \includegraphics[scale=0.25]{Immagini/UMLris_v1_0.png}
        
    \section{Dizionario delle classi}
        
    \section{Dizionario delle associazioni}
    \chapter{Progettazione della Soluzione}

\section{Class Diagram del dominio della soluzione}
\includegraphics[scale=0.05, center]{Immagini/PDiagram.png}

\subsection{Dizionario dei Metodi}
\subsubsection*{Controller}



\section{Sequence Diagram del metodo addArticoloConferenza}
\includegraphics[scale=0.15, center]{Immagini/AddArtConf_SD.png}

\section{Sequence Diagram del metodo checPermessiNotifiche}
\includegraphics[scale=0.28, center]{Immagini/checkPermNot_SD.jpg}

    \chapter{Features e Controlli}


\section{Pagina di Login}
\includegraphics[scale=0.45]{Immagini/Schermate/Login_Register/LoginPage.png}


\section{Pagina di Registrazione}
\includegraphics[scale=0.45]{Immagini/Schermate/Login_Register/RegisterPage.png}

\section{Pagina di Ricerca}
\includegraphics[scale=0.45]{Immagini/Schermate/Search/SearchPage.png}
\subsection{Filtri Libri}
\includegraphics[scale=0.45]{Immagini/Schermate/Search/SearchPage-FiltriLibro.png}
\subsection{Filtri Serie}
\includegraphics[scale=0.45]{Immagini/Schermate/Search/SearchPage-FiltriSerie.png}
\subsection{Filtri Articoli}
\includegraphics[scale=0.45]{Immagini/Schermate/Search/SearchPage-FiltriArticoli.png}
\subsection{Filtri Riviste}
\includegraphics[scale=0.45]{Immagini/Schermate/Search/SearchPage-FiltriRiviste.png}

\section{Pagina del Risultato della Ricerca}
\includegraphics[scale=0.45]{Immagini/Schermate/Search/ResultPage.png}

\section{Pagina di Inserimento Risorsa}
\includegraphics[scale=0.45]{Immagini/Schermate/Insert/InserisciRisorsaPage.png}

\subsection{Inserimento di un Libro}
\includegraphics[scale=0.45]{Immagini/Schermate/Insert/InserisciRisorsaPage-Libro.png}
\subsection{Inserimento di un Libro e della relativa Serie}
\includegraphics[scale=0.45]{Immagini/Schermate/Insert/InserisciRisorsaPage-LibroSerie.png}

\subsection{Inserimento di un Articolo}
\includegraphics[scale=0.45]{Immagini/Schermate/Insert/InserisciRisorsaPage-Articolo.png}
\subsection{Inserimento di un Articolo e della relativa Rivista}
\includegraphics[scale=0.45]{Immagini/Schermate/Insert/InserisciRisorsaPage-ArticoloRivista.png}
\subsection{Inserimento di un Articolo e della relativa Conferenza}
\includegraphics[scale=0.45]{Immagini/Schermate/Insert/InserisciRisorsaPage-ArticoliConferenza.png}

\section{Menù Utente}
\subsection{Notifiche}
\includegraphics[scale=0.45]{Immagini/Schermate/Utente/NotificheUtente.png}
\subsection{Richiesta di disponibilit\'a Serie}
\includegraphics[scale=0.450]{Immagini/Schermate/Utente/RichiediDisponibilita.png}



\end{sloppypar}
\end{document}