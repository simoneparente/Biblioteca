\chapter{Features e Controlli}
 \section{Spiegazione}
 L'applicativo permette ad un utente, dopo un login, di cercare le risorse nella biblioteca online.
 La ricerca pu\'o essere effettuata per nome della risorsa o in maniera più specifica tramite una serie di
 filtri spuntabili.
 L'applicazione, inoltre, da la possibilit\'a ad utente con permessi di amministratore di aggiundere risorse 
 al database.


 \subsection{Finestra di Login}
 Nella \textit{Finesta di Login} l'utente pu\'o decidere di effettuare un login per accedere all'applicativo oppure,
 in caso non avesse un account, di registrarsi cliccando sul link \textit{"Non hai un account? Clicca
 qui per registrarti"}.
 \\
 \includegraphics[scale=0.25, center]{Immagini/Schermate/Login_Register/LoginPage.png}

 \subsection{Finestra di Registrazione Utente}
 Nella \textit{Finestra di Registrazione} l'utente per creare un nuovo account deve riempire i campi: "Username", 
 "Password", "Conferma Password".
 Qualora un account con quello Username dovesse gi\'a esistere, oppure le due Password non dovessero corrispondere, 
 l'appilcativo segnaler\'a l'errore colorando il bordo dei Jtextfield di rosso.
 \\
 \includegraphics[scale=0.25, center]{Immagini/Schermate/Login_Register/RegisterPage.png}

 \subsection{Finestra di Ricerca}
 Nella \textit{Finesta di Ricerca} sar\'a possibile cercare una risorsa nel database, inserendone il titolo nel Jtextfield e 
 selezionandone il tipo. L'utente potr\'a anche scremare i risultati della sua rierca utilizzando, per ogni tipo di risorsa, l'apposita Sezione Filtri.
 La Sezione Filtri \'e composta da dei Jcombobox attivabili tramite Jcheckbox differenti in base alla risorsa da ritrovare.
 \\
 \includegraphics[scale=0.25, center]{Immagini/Schermate/Search/SearchPage.png}

 \subsubsection{Filtri Libri}
 \includegraphics[scale=0.25, center]{Immagini/Schermate/Search/SearchPage-FiltriLibro.png}
 \subsubsection{Filtri Serie}
 \includegraphics[scale=0.25, center]{Immagini/Schermate/Search/SearchPage-FiltriSerie.png}
 \subsubsection{Filtri Articoli}
 \includegraphics[scale=0.25, center]{Immagini/Schermate/Search/SearchPage-FiltriArticoli.png}
 \subsubsection{Filtri Riviste}
 \includegraphics[scale=0.25, center]{Immagini/Schermate/Search/SearchPage-FiltriRiviste.png}

 \subsection{Finestra dei Risultati}
 Nella \textit{Finestra dei Risultati} \'e possibile trovare una JTable con tutti i risultati della query.
 \\
 \includegraphics[scale=0.25, center]{Immagini/Schermate/Search/ResultPage.png}

 \subsection{InsertPage}
 Nella \textit{Finestra di Inserimento} un utente con permessi amministrativi pu\'o inserire nuove risorse nel database tramite gli
 appositi Jtextfield.
 \\
 \includegraphics[scale=0.25, center]{Immagini/Schermate/Insert/InserisciRisorsaPage.png}

 \subsubsection{Libro}
 \includegraphics[scale=0.25, center]{Immagini/Schermate/Insert/InserisciRisorsaPage-Libro.png}
 \subsubsection{Libro in Serie}
 \includegraphics[scale=0.25, center]{Immagini/Schermate/Insert/InserisciRisorsaPage-LibroSerie.png}

 \subsubsection{Articolo}
 \includegraphics[scale=0.25, center]{Immagini/Schermate/Insert/InserisciRisorsaPage-Articolo.png}
 \subsubsection{Articolo in Rivista}
 \includegraphics[scale=0.25, center]{Immagini/Schermate/Insert/InserisciRisorsaPage-ArticoloRivista.png}
 \subsubsection{Articolo in Conferenza}
 \includegraphics[scale=0.25, center]{Immagini/Schermate/Insert/InserisciRisorsaPage-ArticoliConferenza.png}

 \subsection{Utente}
 Nella \textit{Finestra Utente} \'e possibile cambiare password oppure vedere tutte le notifiche ricevute e richiedere la disponibilt\'a per l'acquisto di una serie.
 \subsubsection{Notifiche}
 \includegraphics[scale=0.25, center]{Immagini/Schermate/Utente/NotificheUtente.png}
 \subsubsection{Richiesta di disponibilit\'a Serie}
 \includegraphics[scale=0.50, center]{Immagini/Schermate/Utente/RichiediDisponibilita.png}