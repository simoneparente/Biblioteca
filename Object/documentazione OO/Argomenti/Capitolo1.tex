\chapter{Descrizione e Analisi del Progetto}
\section{Obiettivo}
Si vuole sviluppare un applicativo in linguaggio Java dotato di GUI Swing che 
consenta di gestire una Biblioteca Digitale.
\section{Analisi del problema}
Si svilupper\`a una piattaforma di nome (\textbf{BiblIOTech}) che permette a degli utenti,
previa registrazione, di accedere ai servizi offerti dall'applicazione.
\\ \indent La \textbf{Registrazione} \`e effettuabile tramite l'inserimento di dati
generici come Username e Password, il livello di autorizzazione per i nuovi utenti sar\`a
il pi\`u basso (livello 0), che offrir\`a delle possibilit\`a di gestione limitate 
(per esempio l'aggiunta di una risorsa al database non sar\`a possibile per utenti di livello 0).
\\ \indent Il \textbf{Login} permette l'accesso alla piattaforma agli utenti gi\`a registrati,
 tramite Username e Password.
\\ \indent Le \textbf{Risorse}: Libri, Serie letterarie, Articoli Scientifici e Riviste, 
potranno essere visionate e aggiunte a un database.
\\ \indent Per i \textbf{Libri} devono essere specificati il \emph{titolo}, l' \emph{ISBN}, la
\emph{data di publicazione}, l' \emph{autore (o gli autori)}, l' \emph{editore}, la 
\emph{modalit\`a di fruizione (cartaceo, digitale o audiolibro)} e la \emph{sala/libreria} 
ove il libro sia stato eventualmente presentato.\\ 

Un libro pu\`o avere anche uno o pi\`u seguiti. In tal caso, \`e 
importante prevedere un'interrogazione che permetta di recuperare tutte le librerie dalle quali
sia possibile acquistare l'intera \textbf{serie} dei libri.
\\
Non appena una serie sar\`a disponibile per l'acquisto in almeno una libreria, il sistema 
invier\`a una notifica all'utente che, accedendo alla propria area personale, potr\`a visualizzare
i nomi delle librerie da cui acquistare la serie.
\\ \indent Per gli \textbf{Articoli Scientifici} devono essere specificati il \emph{titolo}, 
il \emph{DOI}, la \emph{data di publicazione}, l' \emph{autore (o autori)}, l' \emph{editore}, 
in quale \textbf{rivista} \emph{(nome, argomento, anno di pubblicazione, responsabile della 
rivista)} o in quale \textbf{conferenza} \emph{(luogo della conferenza, data di inizio e data 
fine, struttura organizzatrice e responsabile della conferenza)} \`e stato pubblicato.
\section{Individuazione delle Responsabilit\`a}

\begin{longtable}[c]{|l|l|}
  \hline
  \textbf{Classe} & \textbf{Responsabilit\`a}                                                                                                    \\ \hline
  \endfirsthead
  %
  \endhead
  %
  Articolo        & Eredita da Pubblicazione.                                                                                                  \\ \hline
  Autore &
    \begin{tabular}[c]{@{}l@{}}Inserimento, modifica e rimozione di \\ un autore e recupero dei dati di \\ quest'ultimo.\end{tabular} \\ \hline
  Conferenza &
    \begin{tabular}[c]{@{}l@{}}Eredita da Evento: inserimento,\\ modifica, rimozione di articoli relativi\\ a una conferenza.\end{tabular} \\ \hline
  Evento          & \begin{tabular}[c]{@{}l@{}}Inserimento di un Evento. Un Evento \`e \\ una  conferenza oppure una presentazione.\end{tabular} \\ \hline
  Libro           & Eredita da Pubblicazione.                                                                                                  \\ \hline
  Negozio         & \begin{tabular}[c]{@{}l@{}}Recupero dati dei Libri acquistabili\\ in un negozio.\end{tabular}                              \\ \hline
  Presentazione   & \begin{tabular}[c]{@{}l@{}}Eredita da Evento. Un Libro può o meno\\ avere una presentazione.\end{tabular}                  \\ \hline
  Pubblicazione &
    \begin{tabular}[c]{@{}l@{}}Inserimento di una pubblicazione e\\ recupero dei dati ad essa associata. \\ Tipi di pubblicazione sono Libri e Articoli\\ scientifici.\end{tabular} \\ \hline
  Richiesta       & \begin{tabular}[c]{@{}l@{}}Inserimento di una richiesta di una\\ specifica Serie da parte di un Utente.\end{tabular}       \\ \hline
  Rivista         & \begin{tabular}[c]{@{}l@{}}Inserimento di una Rivista. Una Rivista \`e \\ un insieme di Articoli scientifici.\end{tabular}   \\ \hline
  Serie           & Inserimento di una Serie.                                                                                                  \\ \hline
  Utente          & \begin{tabular}[c]{@{}l@{}}Registrazione di un Utente alla\\ piattaforma e recupero dei relativi dati.\end{tabular}        \\ \hline
  \end{longtable}