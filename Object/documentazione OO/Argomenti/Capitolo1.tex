\chapter{Descrizione e Analisi del Progetto}
\section{Obiettivo}
Si vuole sviluppare un applicativo in linguaggio java dotato di GUI Swing che 
consenta di gestire una Biblioteca Digitale.
\section{Analisi del problema}
Si svilupper\`a una piattaforma di nome (\textbf{BiblIOTech}) utilizzabile da utenti
che, previa registrazione, potranno accedere ai servizi offerti dall'applicazione.
\\ \indent La \textbf{Registrazione} \`e effettuabile tramite l'inserimento di dati
generici come Username e Password, il livello di autorizzazione per i nuovi utenti sar\`a
il più basso, di default (livello 0), che offrir\`a delle possibilit\`a di gestione limitate 
(per esempio l'aggiunta di una risorsa al database non sar\`a possibile per utenti di livello 0).
\\ \indent Il \textbf{Login} permette l'accesso alla piattaforma agli utenti gi\`a registrati,
 tramite Username e Password.
\\ \indent Le \textbf{Risorse}: Libri, Serie letterarie, Articoli Scientifici e Riviste, 
potranno essere visionate e aggiunte a un database.
\\ \indent Per i \textbf{Libri} devono essere specificati il \emph{titolo}, l' \emph{ISBN}, la
\emph{data di publicazione}, l' \emph{autore (o gli autori)}, l' \emph{editore}, la 
\emph{modalit\`a di fruizione (cartaceo, digitale o audiolibro)} e la \emph{sala/libreria} 
ove il libro sia stato eventualmente presentato.\\ 

Un libro pu\`o avere anche uno o pi\`u seguiti. In tal caso, \`e 
importante prevedere un'interrogazione che permetta di recuperare tutte le librerie dalle quali
sia possibile acquistare l'intera \textbf{serie} dei libri.
\\
Non appena una serie sar\`a disponibile per l'acquisto in almeno una libreria, il sistema 
invier\`a una notifica all'utente che, accedendo alla propria area personale, potr\`a visualizzare
i nomi delle librerie da cui acquistare la serie.
\\ \indent Per gli \textbf{Articoli Scientifici} devono essere specificati il \emph{titolo}, 
il \emph{DOI}, la \emph{data di publicazione}, l' \emph{autore (o autori)}, l' \emph{editore}, 
in quale \textbf{rivista} \emph{(nome, argomento, anno di pubblicazione, responsabile della 
rivista)} o in quale \textbf{conferenza} \emph{(luogo della conferenza, data di inizio e data 
fine, struttura organizzatrice e responsabile della conferenza)} \`e stato pubblicato.
\section{Individuazione delle Responsabilit\'a}